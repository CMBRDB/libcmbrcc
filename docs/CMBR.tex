\documentclass{article}
\usepackage{hyperref}
\usepackage{amsmath}

\title{CMBR Docs}
\author{datawizard}
\date{Specification Version 0.0.1}

\begin{document}

\maketitle

DISCLAIMER: The documentation is still very much so in the making, it is nothing more than a WIP

\section{Definitions}
u24 - 24 bit unsigned integer \\
u8  - 8  bit unsigned integer \\
NAG - Numerical annotation glyph. See \href{https://w.wiki/AWUT}{Wikipedia} for more information. \\
Variation table - A table that denotes  \\

\section{Move notation}
The move is represented with an u24.
\subsection{First part}
\begin{equation}
    0\text{b}\underbrace{000000}_\text{To square} \underbrace{000000}_\text{From square}
    \underbrace{0000}_\text{Piece}\underbrace{00000000}_\text{Flags}
\end{equation}

To and from squares are defined as an index of a chessboard square. The values are between 0-63. Where 0 is 'a1' and 63 is 'h8'.
\par Pieces value represent some piece. See the table for more info
\par The flags are defined as individual values. They're bit-wise ored ($\vert$) together to get the final value.

\begin{center}
\begin{tabular}{|c|c|c|}
 \hline
 \multicolumn{3}{|c|}{CMBR Flag enumeration} \\
 \hline
 Flag name & Binary value & Note \\
 \hline
 FlagNone & 0b00000000 & Empty flag \\
 \hline
 FlagCheck & 0b00000001 & Move is a check \\
 \hline
 FlagMate & 0b00000010 & Move is a checkmate \\
 \hline
 FlagCapture & 0b00000100 & Move is a capture \\
 \hline
 FlagNag & 0b00001000 & If this flag is set \\
  & & The first 8 bits are replaced \\
  & & with a NAG index \\
 \hline
 FlagPromotesBishop & 0b01000000 & Move promotes to bishop \\
 \hline
 FlagPromotesKnight & 0b01010000 & Move promotes to knight \\
 \hline
 FlagPromotesRook & 0b01100000 & Move promotes to rook \\
 \hline
 FlagPromotesQueen & 0b01110000 & Move promotes to queen \\
 \hline
 FlagIsVariationPointer & 0b10000000 & If this flag is set \\
 & & The first 16 bits are replaced \\
 & & with an index to the variations table \\
 \hline
\end{tabular}
\end{center}

\begin{center}
\begin{tabular}{ |c|c| }
 \hline
 \multicolumn{2}{|c|}{Pieces to binary value table} \\
 \hline
 Piece & Binary value \\
 \hline
 White pawn & 0b0000 \\
 \hline
 White knight & 0b0001 \\
 \hline
 White bishop & 0b0010 \\
 \hline
 White rook & 0b0011 \\
 \hline
 White queen & 0b0100 \\
 \hline
 White king & 0b0101 \\
 \hline
 White castles short & 0b0110 \\
 \hline
 White castles long & 0b0111 \\
 \hline
 Black pawn & 0b1000 \\
 \hline
 Black knight & 0b1001 \\
 \hline
 Black bishop & 0b1010 \\
 \hline
 Black rook & 0b1011 \\
 \hline
 Black queen & 0b1100 \\
 \hline
 Black king & 0b1101 \\
 \hline
 Black castles short & 0b1110 \\
 \hline
 Black castles long & 0b1111 \\
 \hline
\end{tabular}
\end{center}

\end{document}
